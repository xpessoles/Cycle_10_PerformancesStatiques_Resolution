\documentclass[10pt,fleqn]{article} % Default font size and left-justified equations
\usepackage[%
    pdftitle={STAT : Mise en oeuvre des démarches de résolution pour déterminer les performances des systèmes en statique},
    pdfauthor={Xavier Pessoles}]{hyperref}
    
\input{style/new_style}
\input{style/macros_SII}

\usepackage{multicol}
\fichetrue
%\fichefalse

\proftrue
%\proffalse

\tdtrue
%\tdfalse

\courstrue
\coursfalse

\def\discipline{Sciences \\Industrielles de \\ l'Ingénieur}
\def\xxtete{Sciences Industrielles de l'Ingénieur}

\def\classe{PTSI}
\def\xxnumpartie{Cycle 10}
\def\xxpartie{Mise en \oe{}uvre des démarches de résolution pour déterminer les performances des systèmes en statique}

\def\xxnumchapitre{Chapitre 1}
\def\xxchapitre{Principe Fondamental de la Statique}

\def\xxtitreexo{Suspension automobile}
\def\xxsourceexo{\hspace{.2cm} D'après ressources de F. Mathurin.}


\def\xxposongletx{2}
\def\xxposonglettext{1.45}
\def\xxposonglety{20}
\def\xxonglet{Cycle 10 -- Ch. 1}

\def\xxactivite{Colle 2}
\def\xxauteur{\textsl{F. Mathurin}}

\def\xxcompetences{%
\textsl{%
\textbf{Savoirs et compétences :}\\
%\noindent \textbf{Analyser :} 
%\begin{itemize}[label=\ding{112},font=\color{ocre}] 
%\item -- %\textit{A3 -- C6 :} transmetteurs de puissance.
%\end{itemize}
%\noindent \textbf{Modéliser :} \textit{proposer un modèle de connaissance du système.}
}}

\def\xxfigures{
\includegraphics[width=.7\textwidth]{images/suspen}
}%figues de la page de garde

\def\xxpied{%
Cycle 10 -- Vérification des performances statiques \\
Ch. 1 : Principe Fondamental de la Statique -- \xxactivite%
}


\setcounter{secnumdepth}{5}
%---------------------------------------------------------------------------


\begin{document}
%\chapterimage{png/Fond_Cin}
\input{style/new_pagegarde}
\vspace{7cm}
\pagestyle{fancy}
\thispagestyle{plain}


\def\columnseprulecolor{\color{ocre}}
\setlength{\columnseprule}{0.4pt} 

\begin{multicols}{2}

On s'intéresse à une suspension automobile dont on donne ci-dessous un extrait de cahier des charges fonctionnel ainsi qu’une modélisation. L'objectif est de vérifier si la suspension satisfait le niveau du critère d'affaissement statique maximal du cahier des charges, c'est à dire vérifier si la voiture, soumise à son propre poids, s'affaisse de moins ou de plus de 12 cm, suite à l'écrasement des amortisseurs. 

\subparagraph{}
\textit{Montrer que $Y_{43} =0$.}

\subparagraph{}
\textit{Déterminer les équations obtenues en appliquant le PFS à l’ensemble \{4+6\} au point $D$.}

\subparagraph{}
\textit{Montrer que $X_{92} =0$.}

\subparagraph{}
\textit{Déterminer les équations obtenues en appliquant le PFS au solide 2 au point $A$.}


\subparagraph{}
\textit{Déterminer toutes les inconnues d'effort en fonction de $F_{06}$.}

Données : $a = 16 \text{cm}$, $b = 33 \text{cm}$, $c = 8 \text{cm}$, $d = 25 \text{cm}$, $h = 3 \text{cm}$, $L = 15 \text{cm}$, $e = 9 \text{cm}$, $\mu = 18 \text{cm}$. 

La raideur du ressort est $k = 100\;000 \text{N/m}$. La masse de la voiture est de 2200 kg.

\subparagraph{}
\textit{Conclure quant à la capacité de la suspension de voiture à satisfaire l’exigence Affaissement statique du cahier des charges. }

\end{multicols}


\begin{center}
\includegraphics[width=.9\linewidth]{images/suspension}
\end{center}


\end{document}


